\documentclass{article}
\usepackage[letterpaper]{geometry}
\usepackage[T1]{fontenc}
\usepackage[spanish]{babel}
\usepackage{listings}

\usepackage{graphicx}
\graphicspath{ {./images} }
\usepackage{float}
\usepackage [autostyle=true, english = american]{csquotes}
\usepackage{enumitem}
\newenvironment{QandA}{\begin{enumerate}\bfseries}
                      {\end{enumerate}}
\newenvironment{answered}{\par\normalfont}{}
\MakeOuterQuote{"}
\setlength{\parskip}{12pt}

\title{Reto - CTF}
\author{
	Monreal De la Rosa, Diego Azael\\
	343427
	\and
	Quistian Navarro, Juan Luis\\
	341807
	\and
	Ing. Sistemas Inteligentes, Gen. 2021\\
	Principios de Seguridad Informática
}
\date{\today}	

\begin{document}
\maketitle

\begin{figure}[H]
  \centering
  \includegraphics[width=0.8\linewidth]{./total.png}
  \caption{}
\end{figure}


\section*{Coding 100}

Para encontrar las palabras, para cada una se checa cada celda de la cuadricula. Si conincide con la primera letra de la palabra, empieza un proceso de busqueda en todas las direcciones. Si la direccion no es valida, queda fuera de la cuadricula, o la letra no coincide con la siguiente de la palabra se termina esa busqueda. Para el patrón de L, si no se ha echo un cambio de dirección y la dirección actual no es diagonal, se checa realiza la busqueda tambien en las direcciones perpendiculares. Si se llega al final de la palabra, se marca cada posición recorrida. Una vez hecho para cada palabra, se eliminan las celdas marcadas y se obtiene la contraseña.

Bandera: \{FLG:c0n9r4t5\_3\_3nj0y\_th3\_4dv3ntur3\}
\begin{figure}[H]
  \centering
  \includegraphics[width=0.5\linewidth]{100.png}
  \caption{}
\end{figure}



\section*{Coding 200}
Bandera: 
\begin{figure}[H]
  \centering
  \includegraphics[width=0.5\linewidth]{200.png}
  \caption{}
\end{figure}
\section*{Coding 300}
Bandera: 
\begin{figure}[H]
  \centering
  \includegraphics[width=0.5\linewidth]{300.png}
  \caption{}
\end{figure}
\section*{Coding 500}
Para poder hacer el interprete, primero se analizaron los ejemplos para determinar el significado de cada expresión.

\begin{itemize}
	\item El lenguaje obtenido fue el siguiente:
	\item Las expresiones son leidas de izquierda a derecha.
	\item Los objetos "STRINGS" comienzan con "B", con cada caracter separado por una 'b', y en orden inverso.
	\item Los objetos "INTEGERS" comienzan con "N" y pueden formarse por una operación aritmética, donde cada elemento se separa por un operador: la 'a' es suma, la 'm' es multiplicación, la 'd' es división, y la 's' es resta.
	\item La operacióñ "PRINT" se indica con una 'P', y se imprime el valor de la siguiente expresión.
	\item Las varaibles se declaran con una 'V', seguido de un nombre de variable, con cada caracter separado por una 'r' y en orden inverso. Si es precedida por un '=', se está asignando el valor, si es cualquier otra expresión, se está evaluando o leyendo.
	\item Las operaciones 'ADD', 'MULT', 'DIV', 'SUB' siempre son seguidas por dos expresiones, y se evaluan de izquierda a derecha.
\end{itemize}

El interprete compienza haciendo un split de la entrada para separar cada expresión, para esto se uso una expresión regular que separara en los cambios a mayuscua, en los espacios, y en las operaciones. Para cada linea se invierte el orden, y se insertan a 
una pila.

Una vez se separan, empieza un ciclo que se repite hasta que la pila esté vacía. En cada iteración se saca un elemento de la pila, si es una operación, se sacan los dos elementos siguientes y se realiza la operación, se guarda el resultado en la pila. Si es una variable, se guarda en un diccionario con el nombre de la variable como llave. Si es un print, se imprime el siguiente elemento.

Para las condiciones, se guarda cada condición individual en una pila y se evalua de izquierda a derecha. Despues se separan las intrucciones para el caso verdadero y la condición alternativa, si es que hay una, en un pila diferente. Una vez separadas, si se cumplio la condición, se iniica un ciclo de ejecución de la pila correspondiente, y sucede lo mismo si se cumple la comdición alterna. Las instrucciones pueden tener condiciones anidadas, por lo que al separar ambas ramas, se lleva un contador para saber en cual nivel de anidación se encuentra.

Los bucles fucnionan de una forma similar, se guarda la condicion en una pila, y se separan las instrucciones a ejecutar. Se crea un copdia de la pila de condicion y se evalua, de ser verdadera, se ejecuta el ciclo, haciendo una copia de las instrucciones a ejecutar, y se repite el proceso hasta que la condición sea falsa. Al tambien haber anidación, tambien se lleva un contador para saber en que nivel de anidación se encuentra.


Bandera: \{FLG:50M371m3Z\_350l4Ng\_c4N\_B3\_M0r3\_R34d4bL3\_7h4N\_1337\_Fl49z\}
\begin{figure}[H]
  \centering
  \includegraphics[width=0.5\linewidth]{500.png}
  \caption{}
\end{figure}
\end{document}
